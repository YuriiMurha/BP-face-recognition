\setcounter{page}{1}
\setcounter{equation}{0}
\setcounter{figure}{0}
\setcounter{table}{0}

\section*{Introduction}
\addcontentsline{toc}{section}{\numberline{}Introduction}
This thesis addresses the problem of face recognition in camera surveillance systems. The motivation for this work stems from the increasing demand for automated attendance and security systems capable of identifying individuals in real-time video streams. The main objective is to design, implement, and evaluate a modular face recognition system that leverages both custom deep learning models and state-of-the-art libraries.

The thesis is structured as follows:
\begin{itemize}
    \item An overview of the tools and libraries used for face recognition, including both custom and pre-built solutions.
    \item A detailed description of the dataset creation, annotation, and augmentation process.
    \item The development and evaluation of deep learning models for face detection and recognition.
    \item Integration of the system components into a real-time attendance application.
    \item Comparative analysis of different face detection and recognition methods.
\end{itemize}

The following chapters provide a comprehensive account of the methods, implementation, and results achieved in this work.

\subsection*{Background}
The pervasive integration of digital technologies into modern society has fundamentally reshaped various sectors, with security and surveillance systems undergoing a particularly transformative evolution. Conventional security paradigms, which are often dependent on manual monitoring and reactive responses, are increasingly inadequate when confronted with the escalating complexity and sophistication of contemporary threats. In response to these challenges, artificial intelligence (AI), particularly in the domain of computer vision, has emerged as a pivotal technology capable of addressing these growing challenges. Facial recognition, a prominent application of computer vision, offers the potential for enhanced automation, efficiency, and accuracy in identifying individuals, thereby bolstering security protocols across diverse environments. This technological shift necessitates a comprehensive understanding of the underlying algorithms, their practical applications, and the inherent ethical and privacy implications.

\subsection*{Motivation}
The motivation for this research stems from the growing demand for intelligent and autonomous security solutions. Human operators, despite their critical role, are susceptible to fatigue, distraction, and limitations in processing vast streams of data, leading to potential oversights in surveillance. AI-driven systems, conversely, offer continuous vigilance and the capacity to analyze large datasets in real-time, identifying anomalies and potential threats with a speed and consistency unattainable by human counterparts~\cite{securityindustry_2025_transforming}. Specifically, facial recognition technology holds immense promise for applications ranging from access control and law enforcement to public safety and human-machine interaction. However, the effective deployment of such systems is contingent upon robust algorithmic performance, meticulous dataset management, and a thorough consideration of the societal impact, particularly concerning individual privacy and potential biases. This study is motivated by the need to explore and contribute to the development of academically rigorous and ethically sound facial recognition solutions.

\subsection*{Problem Statement}
Despite significant advancements in artificial intelligence and computer vision, the development of universally robust, accurate, and ethically compliant facial recognition systems remains a complex challenge. Current systems often face limitations in real-world scenarios due to variations in illumination, facial pose, expression, occlusions, and demographic diversity. Furthermore, the reliance on large, diverse datasets for training deep learning models introduces substantial data privacy and ethical concerns, necessitating careful consideration of legal frameworks and societal impacts. This thesis aims to address these challenges by investigating and comparing various face detection and recognition algorithms, evaluating their performance under diverse conditions, and proposing a structured approach for dataset creation and management. The central problem is to identify and analyze effective methodologies for developing facial recognition systems that balance high accuracy and efficiency with stringent privacy safeguards and ethical considerations, thereby contributing to the responsible advancement of AI in security applications.