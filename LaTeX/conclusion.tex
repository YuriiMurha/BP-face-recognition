\section{Conclusion}

This thesis has explored the development and evaluation of face recognition systems for surveillance applications, addressing the challenges of real-world deployment. The study focused on creating a robust pipeline for dataset creation, preprocessing, model training, and evaluation, leveraging state-of-the-art deep learning techniques.

The proposed system demonstrated significant potential in improving the accuracy and efficiency of face recognition in diverse conditions, including variations in lighting, pose, and occlusions. By employing EfficientNetB0 as the backbone and implementing advanced data augmentation techniques, the model achieved high generalization capabilities. The evaluation of various face detection and recognition algorithms provided valuable insights into their performance, highlighting the trade-offs between accuracy, detection time, and computational complexity.

Despite these advancements, the study acknowledges certain limitations, such as the dependency on high-quality datasets and the computational demands of deep learning models. Future research could explore lightweight architectures and techniques for reducing bias in face recognition systems, ensuring ethical and fair deployment.

In conclusion, this thesis contributes to the field of artificial intelligence and computer vision by presenting a comprehensive approach to face recognition in surveillance systems. The findings underscore the importance of integrating robust algorithms with ethical considerations, paving the way for responsible and effective security solutions.