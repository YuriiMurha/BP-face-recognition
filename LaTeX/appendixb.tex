\section*{Appendix B: Code Samples}
\addcontentsline{toc}{section}{Appendix B: Code Samples}

% Example code sample for face detection (OpenCV)
\begin{verbatim}
# Example Python code for face detection
import cv2
face_cascade = cv2.CascadeClassifier('haarcascade_frontalface_default.xml')
img = cv2.imread('test.jpg')
gray = cv2.cvtColor(img, cv2.COLOR_BGR2GRAY)
faces = face_cascade.detectMultiScale(gray, 1.3, 5)
for (x, y, w, h) in faces:
    cv2.rectangle(img, (x, y), (x+w, y+h), (255, 0, 0), 2)
cv2.imshow('img', img)
cv2.waitKey(0)
cv2.destroyAllWindows()
\end{verbatim}

% Example code sample for dataset splitting
\begin{verbatim}
# Example Python code for splitting dataset into train/val/test
import os, numpy as np
all_images = os.listdir('images')
np.random.shuffle(all_images)
train_count = int(len(all_images) * 0.7)
test_count = int(np.ceil(len(all_images) * 0.15))
val_count = len(all_images) - train_count - test_count
train_files = all_images[:train_count]
test_files = all_images[train_count:train_count+test_count]
val_files = all_images[train_count+test_count:]
# Move files to appropriate folders as needed
\end{verbatim}

% Example citation of code in thesis:
% See Appendix~\ref{appendixb} for code samples used in the implementation~\cite{opencv_library, numpy_library}.

% Add more code samples as needed, referencing them in the main text.