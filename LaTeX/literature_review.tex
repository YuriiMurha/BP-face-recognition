\section{Literature Review}

\subsection{Real-World Applications}

\subsubsection{Surveillance Systems}

\paragraph{AI in the Security Industry Today}
The contemporary security landscape is increasingly integrating artificial intelligence (AI) to address evolving threats and enhance operational efficiencies beyond traditional methods~\cite{securityindustry_2025_transforming}. AI systems are capable of processing vast quantities of data from diverse sources simultaneously, enabling the identification of patterns and potential threats that human operators might overlook~\cite{securityindustry_2025_transforming}. This capability is particularly valuable for improving the core challenges encountered in modern Global Security Operations Centers (GSOCs)~\cite{securityindustry_2025_transforming}. For instance, AI-driven technologies can intelligently filter and verify alarms by analyzing multiple data points to ascertain the likelihood of a genuine security threat, thereby improving response times and mitigating alarm fatigue~\cite{securityindustry_2025_transforming}. In surveillance operations, AI continuously monitors video feeds, detecting and classifying objects, individuals, and behaviors in real time~\cite{securityindustry_2025_transforming}. Such systems can automatically alert operators to suspicious activities, including loitering, abandoned objects, or unauthorized access attempts, thus significantly expanding the effective coverage area without necessitating additional human resources~\cite{securityindustry_2025_transforming}.

The integration of AI into security operations is not primarily aimed at replacing human personnel but rather at augmenting their capabilities~\cite{securityindustry_2025_transforming}. By automating routine tasks, AI liberates security professionals to concentrate on strategic decision-making and complex threat assessments~\cite{securityindustry_2025_transforming}. This paradigm shift transforms security roles from passive monitoring to active analysis and strategic planning, potentially fostering more engaging career paths and reducing burnout within the industry~\cite{securityindustry_2025_transforming}. Successful AI integration necessitates meticulous planning and implementation, considering technical, human, and operational factors~\cite{securityindustry_2025_transforming}. Organizations that invest in appropriate infrastructure, training, and change management are better positioned to realize the full potential of AI-enabled security operations, transforming physical security for the modern era~\cite{securityindustry_2025_transforming}.

\paragraph{The Definition of AI Security}
AI security encompasses the comprehensive measures undertaken to safeguard artificial intelligence systems from cyberattacks, data breaches, and other security vulnerabilities. Given the increasing ubiquity of AI systems in both commercial and domestic environments, the imperative for robust security protocols to protect these systems has become paramount. Security assessments for AI systems typically extend across three critical dimensions:
\begin{itemize}
    \item \textbf{Software Level:} Ensuring the security of AI software necessitates conventional code analysis, thorough investigation of programming vulnerabilities, and the execution of regular security audits.
    \item \textbf{Learning Level:} Vulnerabilities at the learning level are intrinsic to AI systems. Protection in this dimension involves securing databases, controlling data ingress, and monitoring the model's performance for anomalous behavior.
    \item \textbf{Distributed Level:} For AI models comprising multiple components that process data independently before consolidating results for a final decision, it is crucial to ensure that each instance of the distributed system functions as intended throughout the operational workflow.
\end{itemize}

\paragraph{How Can AI Enhance Security Systems?}
AI can significantly augment traditional security systems by providing capabilities that surpass human limitations in data processing and vigilance. Key enhancements include:
\begin{itemize}
    \item \textbf{Automated Threat Detection and Alerting:} AI-powered video analytics can automatically identify suspicious activities, such as unauthorized access, loitering, or object recognition, and trigger immediate alerts to security personnel~\cite{securityindustry_2025_transforming}. This capability reduces reliance on constant human monitoring, minimizing human error and fatigue.
    \item \textbf{Predictive Analytics:} By analyzing historical data and real-time inputs, AI can predict potential security breaches or anticipate crime patterns, enabling proactive security measures rather than reactive responses~\cite{securityindustry_2025_transforming}.
    \item \textbf{Improved Efficiency and Resource Optimization:} Automation of routine tasks, such as alarm verification and preliminary incident recording, allows security staff to focus on critical incidents and strategic decision-making, optimizing resource allocation~\cite{securityindustry_2025_transforming}. Organizations often experience decreased staffing requirements for routine monitoring and reduced training costs through automated assistance~\cite{securityindustry_2025_transforming}.
    \item \textbf{Enhanced Accuracy and Reduced False Positives:} Intelligent filtering of alarms and analysis of multiple data points by AI systems can significantly reduce false positives, ensuring that human attention is directed towards genuine threats~\cite{securityindustry_2025_transforming}.
    \item \textbf{Scalability:} AI systems can manage an increasing number of surveillance feeds and data points without a proportional increase in human operators, offering substantial scalability advantages for large-scale deployments~\cite{securityindustry_2025_transforming}.
\end{itemize}

\paragraph{Improving Surveillance Efficiency and Accuracy with AI}
The application of AI in surveillance significantly bolsters both efficiency and accuracy. Real-time video analytics, powered by AI, enable continuous monitoring of numerous camera feeds, surpassing the capacity of human observers~\cite{securityindustry_2025_transforming}. AI systems can rapidly detect and classify objects, individuals, and behaviors, automatically identifying anomalies or suspicious activities that might otherwise be missed~\cite{securityindustry_2025_transforming}. For instance, AI can be trained to recognize specific postures indicative of distress or to identify attempts to bypass access controls. This automated vigilance leads to faster detection of incidents and more precise alerting, minimizing response times and reducing the burden on human operators who traditionally manage overwhelming amounts of data. Furthermore, AI's ability to learn from data allows for continuous improvement in accuracy over time, adapting to new patterns and environments.

\paragraph{Key Must-Have Features of Facial Recognition Software}
Effective facial recognition software (FRS) incorporates several critical features to ensure high performance, security, and ethical operation.

\subparagraph{Robust and Diverse Training Data}
The efficacy of any FRS is directly contingent upon the quality and breadth of its training dataset~\cite{kairos_secret_2018, geeksforgeeks_dataset_2025}. An optimal dataset must be continuously expanding and exhibit significant diversity in terms of demographic attributes such as gender, ethnicity, and age~\cite{geeksforgeeks_dataset_2025}. Furthermore, it should encompass a wide variance in lighting conditions, facial poses (angles), and expressions~\cite{kairos_secret_2018}. The inclusion of images at varying resolutions is also vital to enable the system to perform effectively across different input qualities~\cite{geeksforgeeks_dataset_2025}.

\subparagraph{Data Security and User Privacy}
Given the highly sensitive nature of biometric data, such as faceprints, robust security measures and strict adherence to user privacy principles are paramount for FRS~\cite{getfocal_biometric_2025, transcend_ccpa_2025}. This includes the mandatory encryption of user data and its regular purging to prevent unauthorized access or misuse~\cite{getfocal_biometric_2025}. Software providers must also establish comprehensive incident response plans to address potential data breaches effectively~\cite{getfocal_biometric_2025}. Compliance with regulations like the General Data Protection Regulation (GDPR) in Europe and the California Consumer Privacy Act (CCPA)/California Privacy Rights Act (CPRA) in the United States is essential, as these laws classify biometric data as sensitive personal information requiring explicit consent and stringent safeguards~\cite{getfocal_biometric_2025, transcend_ccpa_2025}.

\subparagraph{Algorithmic Accuracy and Performance Metrics}
The primary metrics for evaluating FRS algorithms are the False Acceptance Rate (FAR) and the False Rejection Rate (FRR)~\cite{recfaces_false_2025, kairos_secret_2018}. FAR occurs when the system incorrectly identifies a different individual as a legitimate user, leading to a false positive~\cite{recfaces_false_2025, kairos_secret_2018}. A low FAR is critical in security applications to prevent unauthorized access~\cite{recfaces_false_2025}. FRR happens when the system fails to recognize an authorized user, resulting in a false negative~\cite{recfaces_false_2025, kairos_secret_2018}. While a high FRR can negatively impact user convenience, there is an inherent trade-off between FAR and FRR; lowering one typically increases the other~\cite{recfaces_false_2025, kairos_secret_2018}. The Equal Error Rate (EER) represents the point where FAR and FRR are equal, serving as a common indicator of overall system accuracy, with lower EER values signifying higher accuracy~\cite{recfaces_false_2025, kairos_secret_2018}. Achieving optimal performance requires precise feature extraction, as overall system accuracy is not solely dependent on the biometric algorithm~\cite{kairos_secret_2018}.

\subparagraph{Scalability}
For large-scale deployments, such as enterprise authentication across multiple locations, the scalability of the FRS is a crucial consideration~\cite{securityindustry_2025_transforming}. The software must be capable of efficiently handling an expanding user base and increasing data volumes.

\subparagraph{Adaptability and Support}
FRS providers should offer robust fallback mechanisms in case of system failures, potentially requiring human oversight and support to maintain operations~\cite{securityindustry_2025_transforming}. Comprehensive support for hardware setup, especially camera calibration, is also vital to maximize accuracy and system effectiveness.

\subparagraph{Transparency and Ethics}
The deployment of FRS has faced significant scrutiny regarding transparency and ethical implications~\cite{ergun_2025_ethical, sustainability_2025_ethical}. It is imperative that the software operates transparently and adheres to strict ethical guidelines, avoiding practices such as unethical data collection (e.g., social media scraping for training data) or privacy violations~\cite{ergun_2025_ethical, sustainability_2025_ethical}. Concerns about algorithmic bias, particularly in terms of accuracy across different demographic groups, and the potential for a "chilling effect" on free speech also highlight the need for ethical implementation and oversight~\cite{sustainability_2025_ethical}.

\paragraph{Future of Artificial Intelligence in the Security Industry}
The influence of artificial intelligence on the physical security industry is projected to expand significantly~\cite{securityindustry_2025_transforming}. AI's inherent capacity to intelligently link and analyze vast amounts of data, derive independent conclusions, and automate predictions presents unprecedented opportunities~\cite{securityindustry_2025_transforming}. This is particularly pertinent in the security sector, where large datasets necessitate meaningful processing~\cite{securityindustry_2025_transforming}.

Deep-learning technologies, in particular, are anticipated to yield unparalleled insights into human behavior, enabling video surveillance systems to monitor and predict criminal activity with enhanced precision~\cite{securityindustry_2025_transforming}. This forward-looking capability facilitates a shift towards more proactive security strategies. Furthermore, AI is expected to continue its growth in delivering scalable solutions across a diverse array of vertical markets, further solidifying its integral role in future security paradigms~\cite{securityindustry_2025_transforming}. Regulatory frameworks, such as the upcoming EU AI Act, will increasingly influence the deployment of such technologies, emphasizing aspects like human oversight and privacy impact assessments for high-risk applications~\cite{getfocal_biometric_2025}.

\subsubsection{Human-Machine Interaction}
The increasing integration of AI into security and surveillance systems necessitates a critical examination of Human-Machine Interaction (HMI) principles. Effective HMI design is crucial for ensuring that human operators can efficiently and reliably interact with complex AI-driven systems. This involves optimizing interfaces for clarity, interpretability, and control, particularly when AI systems are making autonomous decisions or providing alerts that require human verification. Research in this area focuses on developing intuitive dashboards for real-time monitoring, designing effective alert systems that reduce false alarms while highlighting critical events, and building trust in AI capabilities without fostering over-reliance~\cite{securityindustry_2025_transforming}. Challenges include managing cognitive load for operators, mitigating the impact of algorithmic bias on human decision-making, and ensuring transparency in AI's reasoning processes to facilitate human understanding and intervention when necessary. The goal is to create a symbiotic relationship where AI enhances human capabilities, rather than replacing them, leading to more robust and adaptable security operations.

\section{Datasets}

\subsection{Importance of High-Quality Datasets}
The efficacy of facial recognition systems is fundamentally dependent on the quality, diversity, and representativeness of the datasets used for training and evaluation~\cite{kairos_secret_2018, geeksforgeeks_dataset_2025}. High-quality datasets must encompass a broad spectrum of demographic attributes, lighting conditions, facial poses, and expressions to ensure robust generalization and minimize algorithmic bias. The inclusion of images at varying resolutions and from diverse sources further enhances the system's ability to perform effectively in real-world scenarios.

\subsection{Evaluation of Pre-trained Models on Custom Dataset}
To assess the generalizability and robustness of pre-trained facial recognition models, it is essential to evaluate their performance on custom datasets that reflect the target deployment environment. This process involves benchmarking models against metrics such as accuracy, false acceptance rate (FAR), and false rejection rate (FRR), and analyzing their sensitivity to variations in data quality and distribution. The results inform the selection and fine-tuning of models for specific applications, ensuring optimal performance and fairness.