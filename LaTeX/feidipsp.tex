% !TeX encoding=utf-8
\documentclass[a4paper]{feidipsp}
\usepackage[pdftex]{graphicx}
\DeclareGraphicsExtensions{.pdf}
\graphicspath{{figures/}}

\usepackage[english]{babel}
\usepackage[utf8]{inputenc}
\usepackage[T1]{fontenc}
\usepackage{lmodern}

\usepackage{amsmath,amsfonts,amssymb,latexsym}

\def\figurename{Figure}
\def\tabname{Table}

%\usepackage[dvips]{graphicx}
%\DeclareGraphicsExtensions{.eps}

%\usepackage[pdftex]{hyperref}   %% tlac !!!
\usepackage[pdftex,colorlinks,citecolor=magenta,bookmarksnumbered,unicode,pdftoolbar=true,pdfmenubar=true,pdfwindowui=true,bookmarksopen=true]{hyperref}
\hypersetup{%
baseurl={http://www.tuke.sk/sevcovic},
pdfcreator={pdfcsLaTeX},
pdfkeywords={System manual},
pdftitle={Face Detection using Camera Footage},
pdfauthor={Yurii Murha},
pdfsubject={Bachelor Thesis}
}

% Citovanie podla mena autora a roku
%\usepackage[numbers]{natbib}
\usepackage{natbib} \citestyle{chicago}

%\usepackage{mathptm} %\usepackage{times}
\department{Department of Cybernetics and Artificial Intelligence}
\odbor{Computer Science}
\autor{Yurii Murha}
\veduci{Ing.~Jan~Magyar,~PhD}
\konzultant{Ing.~Jan~Magyar,~PhD}
\nazov{Face Recognition Using Camera Footage}
\kratkynazov{Face Recognition}
\nazovprogramu{FaceClassifier}
\keywords{Face Recognition, Python, OpenCV, Neural Network, TensorFlow, Face Detection}
\title{Face Detection using Camera Footage}
\keywords{Manipulator, Matlab, Simulink, Dynamics, kinematics, Control of dynamic systems}
\datum{30.~05.~2025}



\begin{document}
\bibliographystyle{dcu}

\titulnastrana

\tableofcontents

\newpage

\setcounter{page}{1}

\section{System Manual}

This system manual provides detailed technical information about the face recognition project, including libraries, techniques, and methodologies used.

\subsection{Libraries and Dependencies}

The project relies on the following libraries and tools:

\begin{itemize}
    \item \textbf{TensorFlow (v2.19.0):} Used for building and training deep learning models.
    \item \textbf{OpenCV (v4.11.0):} Used for image processing and face detection.
    \item \textbf{Albumentations (v2.0.7):} Used for data augmentation.
    \item \textbf{dlib (v19.24.6):} Provides face detection and recognition capabilities.
    \item \textbf{Matplotlib (v3.10.1):} Used for visualizing data and results.
    \item \textbf{Pandas (v2.2.3):} Used for data manipulation and analysis.
    \item \textbf{TQDM (v4.67.1):} Used for progress bars during processing.
    \item \textbf{Shapely (v2.1.0):} Used for calculating Intersection over Union (IoU) for bounding boxes.
\end{itemize}

Refer to the \texttt{requirements.txt} file for a complete list of dependencies and their versions.

\subsection{Techniques and Methodologies}

\subsubsection{1. Data Collection and Annotation}
\begin{itemize}
    \item Images are collected using OpenCV and saved in the \texttt{data/datasets/images} directory.
    \item Annotation is performed using the Labelme tool, which generates JSON files containing bounding box information.
\end{itemize}

\subsubsection{2. Data Preprocessing and Augmentation}
\begin{itemize}
    \item Images are resized and normalized to ensure consistency.
    \item Augmentation techniques include random cropping, flipping, brightness adjustment, and gamma correction.
    \item Bounding boxes are adjusted to match augmented images.
\end{itemize}

\subsubsection{3. Face Detection and Recognition}
\begin{itemize}
    \item \textbf{Haar Cascades:} Used for real-time face detection based on edge detection.
    \item \textbf{Histogram of Oriented Gradients (HOG):} Used for detecting facial structures.
    \item \textbf{Convolutional Neural Networks (CNNs):} Used for feature extraction and classification.
    \item \textbf{FaceNet:} Used for generating embeddings for face recognition.
\end{itemize}

\subsubsection{4. Model Training}
\begin{itemize}
    \item The model is built using TensorFlow's Functional API.
    \item EfficientNetB0 is used as the backbone for feature extraction.
    \item The model is trained using a combination of classification and regression losses.
    \item Learning rate scheduling and dropout are used to prevent overfitting.
\end{itemize}

\subsubsection{5. Evaluation Metrics}
\begin{itemize}
    \item \textbf{Accuracy:} Measures the percentage of correctly detected faces.
    \item \textbf{False Positives and False Negatives:} Evaluates the reliability of face detection.
    \item \textbf{Intersection over Union (IoU):} Measures the overlap between predicted and ground truth bounding boxes.
\end{itemize}

\subsection{System Architecture}

The system consists of the following components:
\begin{enumerate}
    \item \textbf{Data Collection:} Captures and annotates images.
    \item \textbf{Preprocessing:} Normalizes and augments the dataset.
    \item \textbf{Model Training:} Builds and trains a deep learning model.
    \item \textbf{Evaluation:} Compares the performance of different face detection methods.
\end{enumerate}

\subsection{Conclusion}

This system manual provides a comprehensive overview of the technical aspects of the project. For further details, refer to the thesis documentation or contact the project maintainer.

\newpage
\def\refname{References}
\addcontentsline{toc}{section}{\numberline{}References}

\begin{thebibliography}{999}

\end{thebibliography}

\addcontentsline{toc}{section}{\numberline{}List of Figures}
\listoffigures

\addcontentsline{toc}{section}{\numberline{}List of Tables}
\listoftables

\end{document}
