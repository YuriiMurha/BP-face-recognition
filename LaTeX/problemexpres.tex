%%
\section{The Problem Expression}

The bachelor thesis focuses on the development and evaluation of a face recognition system for surveillance applications. The primary objective is to address the challenges associated with real-world face detection and recognition, such as variations in lighting, facial pose, occlusions, and demographic diversity. The thesis aims to explore and implement robust methodologies for dataset creation, preprocessing, model training, and evaluation.

The tasks addressed in this thesis include:
\begin{itemize}
    \item Designing and implementing a pipeline for dataset creation and annotation, ensuring diversity and quality in the collected data.
    \item Developing preprocessing techniques, including data augmentation, to enhance model robustness and generalization.
    \item Building a deep learning model using state-of-the-art architectures like EfficientNetB0, capable of detecting and recognizing faces in diverse conditions.
    \item Evaluating the performance of various face detection and recognition algorithms using custom datasets and benchmarking metrics such as accuracy, detection time, and false positives.
    \item Integrating the developed system into a modular architecture for real-time surveillance applications.
\end{itemize}

The thesis also considers the ethical and privacy implications of deploying face recognition systems, ensuring compliance with legal frameworks and societal expectations. The proposed solutions are evaluated under diverse conditions to validate their effectiveness and reliability.